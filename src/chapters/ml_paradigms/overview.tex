\section{Overview}

\subsection*{Introduction}
Machine Learning paradigms (also known as Types of Machine Learning) are approaches or frameworks used to design and train ML models. They define how models learn from data and make predictions or decisions.

\subsection*{Key Considerations in ML Paradigms}
\begin{enumerate}
    \item \textbf{Data Availability}: The choice of paradigm often depends on the availability of labeled or unlabeled data.
    \item \textbf{Computational Resources}: Certain paradigms (e.g., deep reinforcement learning) are computationally intensive.
    \item \textbf{Domain Knowledge}: Understanding the domain can help select the most suitable paradigm and tailor the learning process.
    \item \textbf{Evaluation Metrics}: Different paradigms may require distinct evaluation strategies, such as precision-recall for supervised tasks or silhouette scores for clustering.
\end{enumerate}

\subsection*{Conclusion}
By understanding these paradigms and their characteristics, practitioners can choose the appropriate approach to tackle specific problems, ensuring efficient and effective ML model development.
